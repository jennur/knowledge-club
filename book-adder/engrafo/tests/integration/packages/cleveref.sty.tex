% based on https://texblog.org/2013/05/06/cleveref-a-clever-way-to-reference-in-latex/
\documentclass[11pt]{article}
\usepackage{graphicx}
\usepackage{amsmath}
\usepackage{cleveref}
\begin{document}

\begin{align}
y&=a_1x+b_1\label{eqn:1}
\end{align}

\noindent
Standard equation reference (\textbackslash ref): \ref{eqn:1}\\
Cleveref equation reference (\textbackslash cref): \cref{eqn:1}

\begin{figure}[ht]\centering\rule{0.5\linewidth}{0.1\linewidth}\caption{First figure}\label{fig:1}\end{figure}

\noindent
Standard figure reference (\textbackslash ref): \ref{fig:1}\\
Cleveref figure reference (\textbackslash cref): \cref{fig:1}

\begin{align}
y&=a_1x+b_1\label{eqn:1}\\
y&=a_2x+b_2\label{eqn:2}\\
y&=a_3x+b_3\label{eqn:3}\\
y&=a_4x+b_4\label{eqn:4}
\end{align}

\noindent
Range example: \crefrange{eqn:1}{eqn:4}

\begin{figure}[ht]\centering\rule{0.5\linewidth}{0.1\linewidth}\caption{First figure}\label{fig:1}\end{figure}

\noindent
Mixed references example: \cref{eqn:1,eqn:3,eqn:4,fig:1}

\end{document}
