\documentclass{article}
\usepackage{multirow}
\usepackage{tabulary}
\usepackage{tabularx}
\begin{document}
  % https://en.wikibooks.org/wiki/LaTeX/Tables

  \section{Basic}

  \begin{tabular}{ l c r }
    1 & 2 & 3 \\
    4 & 5 & 6 \\
    7 & 8 & 9 \\
  \end{tabular}

  \section{Vertical lines}

  \begin{tabular}{ l | c | r }
    1 & 2 & 3 \\
    4 & 5 & 6 \\
    7 & 8 & 9 \\
  \end{tabular}

  \section{Horizontal lines at top and bottom}

  \begin{tabular}{ l | c | r }
    \hline
    1 & 2 & 3 \\
    4 & 5 & 6 \\
    7 & 8 & 9 \\
    \hline
  \end{tabular}

  \section{Lines between every row}

  \begin{center}
    \begin{tabular}{ l | c | r }
      \hline
      1 & 2 & 3 \\ \hline
      4 & 5 & 6 \\ \hline
      7 & 8 & 9 \\
      \hline
    \end{tabular}
  \end{center}

  \begin{tabular}{|r|l|}
    \hline
    7C0 & hexadecimal \\
    3700 & octal \\ \cline{2-2}
    11111000000 & binary \\
    \hline \hline
    1984 & decimal \\
    \hline
  \end{tabular}

  \section{Text column with width}

  \begin{center}
    \begin{tabular}{ | l | l | l | p{5cm} |}
      \hline
      Day & Min Temp & Max Temp & Summary \\ \hline
      Monday & 11C & 22C & A clear day with lots of sunshine.
      However, the strong breeze will bring down the temperatures. \\ \hline
      Tuesday & 9C & 19C & Cloudy with rain, across many northern regions. Clear spells
      across most of Scotland and Northern Ireland,
      but rain reaching the far northwest. \\ \hline
      Wednesday & 10C & 21C & Rain will still linger for the morning.
      Conditions will improve by early afternoon and continue
      throughout the evening. \\
      \hline
    \end{tabular}
  \end{center}

  \section{Defining multiple columns}

  \begin{tabular}{l*{6}{c}r}
    Team              & P & W & D & L & F  & A & Pts \\
    \hline
    Manchester United & 6 & 4 & 0 & 2 & 10 & 5 & 12  \\
    Celtic            & 6 & 3 & 0 & 3 &  8 & 9 &  9  \\
    Benfica           & 6 & 2 & 1 & 3 &  7 & 8 &  7  \\
    FC Copenhagen     & 6 & 2 & 1 & 3 &  5 & 8 &  7  \\
  \end{tabular}

  \section{Rows spanning multiple columns}

  \begin{tabular}{ |l|l| }
    \hline
    \multicolumn{2}{|c|}{Team sheet} \\
    \hline
    GK & Paul Robinson \\
    LB & Lucas Radebe \\
    DC & Michael Duberry \\
    DC & Dominic Matteo \\
    RB & Dider Domi \\
    MC & David Batty \\
    MC & Eirik Bakke \\
    MC & Jody Morris \\
    FW & Jamie McMaster \\
    ST & Alan Smith \\
    ST & Mark Viduka \\
    \hline
  \end{tabular}

  \section{Columns spanning multiple rows}

  \begin{tabular}{ |l|l|l| }
    \hline
    \multicolumn{3}{ |c| }{Team sheet} \\
    \hline
    Goalkeeper & GK & Paul Robinson \\ \hline
    \multirow{4}{*}{Defenders} & LB & Lucas Radebe \\
     & DC & Michael Duburry \\
     & DC & Dominic Matteo \\
     & RB & Didier Domi \\ \hline
    \multirow{3}{*}{Midfielders} & MC & David Batty \\
     & MC & Eirik Bakke \\
     & MC & Jody Morris \\ \hline
    Forward & FW & Jamie McMaster \\ \hline
    \multirow{2}{*}{Strikers} & ST & Alan Smith \\
     & ST & Mark Viduka \\
    \hline
  \end{tabular}

  \begin{tabular}{cc|c|c|c|c|l}
    \cline{3-6}
    & & \multicolumn{4}{ c| }{Primes} \\ \cline{3-6}
    & & 2 & 3 & 5 & 7 \\ \cline{1-6}
    \multicolumn{1}{ |c  }{\multirow{2}{*}{Powers} } &
    \multicolumn{1}{ |c| }{504} & 3 & 2 & 0 & 1 &     \\ \cline{2-6}
    \multicolumn{1}{ |c  }{}                        &
    \multicolumn{1}{ |c| }{540} & 2 & 3 & 1 & 0 &     \\ \cline{1-6}
    \multicolumn{1}{ |c  }{\multirow{2}{*}{Powers} } &
    \multicolumn{1}{ |c| }{gcd} & 2 & 2 & 0 & 0 & min \\ \cline{2-6}
    \multicolumn{1}{ |c  }{}                        &
    \multicolumn{1}{ |c| }{lcm} & 3 & 3 & 1 & 1 & max \\ \cline{1-6}
  \end{tabular}

  \begin{tabular}{ r|c|c| }
    \multicolumn{1}{r}{}
     &  \multicolumn{1}{c}{noninteractive}
     & \multicolumn{1}{c}{interactive} \\
    \cline{2-3}
    massively multiple & Library & University \\
    \cline{2-3}
    one-to-one & Book & Tutor \\
    \cline{2-3}
  \end{tabular}

  \section{Table with width}

  \begin{tabular*}{0.75\textwidth}{@{\extracolsep{\fill} } | c | c | c | r | }
    \hline
    label 1 & label 2 & label 3 & label 4 \\
    \hline
    item 1  & item 2  & item 3  & item 4  \\
    \hline
  \end{tabular*}

  \section{tabularx}

  \begin{tabularx}{\textwidth}{ |X|X|X|X| }
    \hline
    label 1 & label 2 & label 3 & label 4 \\
    \hline
    item 1  & item 2  & item 3  & item 4  \\
    \hline
  \end{tabularx}

  \section{tabulary}

  \begin{center}
    \begin{tabulary}{0.7\textwidth}{LCL}
      Short sentences      & \#  & Long sentences                                                 \\
      \hline
      This is short.       & 173 & This is much loooooooonger, because there are many more words.  \\
      This is not shorter. & 317 & This is still loooooooonger, because there are many more words. \\
    \end{tabulary}
  \end{center}

  \section{Captions and references}

  \begin{table}[h]
    \begin{tabular}{ c c c }
       cell1 & cell2 & cell3 \\
       cell4 & cell5 & cell6 \\
       cell7 & cell8 & cell9
    \end{tabular}
    \caption{Some numbers.}
    \label{table:1}
  \end{table}

  Take a look at some cool data in table \ref{table:1}.

  \begin{table}[h]
    \caption{Caption at the top is put at the bottom.}
    \begin{tabular}{ c c c }
       cell7 & cell8 & cell9
    \end{tabular}
  \end{table}
\end{document}
